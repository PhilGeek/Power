%!TEX TS-program = xelatex 
%!TEX TS-options = -output-driver="xdvipdfmx -q -E"
%!TEX encoding = UTF-8 Unicode
%
%  power.tex
%
%  Created by Mark Eli Kalderon on 2011-11-08.
%  Copyright (c) 2011. All rights reserved.
%

\documentclass[12pt]{article} 

% Definitions
\newcommand\mykeywords{disjunctivism, perception, capacity} 
\newcommand\myauthor{Mark Eli Kalderon} 
\newcommand\mytitle{Experiential Pluralism and the Power of Perception}

\input{preamble/preamble}

\usepackage{enumerate}

%%% BEGIN DOCUMENT
\begin{document}

% Title Page
\maketitle
% \begin{abstract} % optional
% \end{abstract} 
\vskip 2em \hrule height 0.4pt \vskip 2em
% \epigraph{text of epigraph}{\textsc{author of epigraph}} % optional; make sure to uncomment \usepackage{epigraph}

% Layout Settings
\setlength{\parindent}{1em}

% Main Content

\section{Introduction} % (fold)
\label{sec:introduction}

Sight is for the sake of seeing. 

% section introduction (end)

\section{The Stuff-Happens Model} % (fold)
\label{sec:the_stuff_happens_model}

In `The Inward Turn', Travis introduces the Stuff-Happens model as follows:
\begin{quote}
    No one, I think, thinks there is any sense of ‘see’ in which merely to have something `affect your eyes'---say, form retinal images---is to see it. Why not? A natural, though not inevitable, idea is: before one saw anything, more stuff would have to happen. (Not inevitable: to have images on one’s retinas would not be to see something whether or not more stuff needed to happen.) A further natural idea might then be: at a certain point the relevant stuff, or enough of it, has happened. At that point, the perceiver goes into a certain `internal' state, the upshot of the stuff, where this is one of a specified range of states into which a particular device, `that which enables vision', might go. In such a state, the idea is, one enjoys visual awareness (or experience). One sees only in enjoying visual awareness. The state decides what visually awareness one thus enjoys; thus, at least, what it is in which one might be seeing something. \citep[315]{Travis:2009fk}
\end{quote}
The Stuff-Happens model is widely subscribed to. \citet{Evans:1982ly} and \citet{Burge:2010uq} count among its advocates. However, when initially confronted with Travis' description of the model, I immediately felt that something important had been left out of account. I now recognize what that is. The Stuff-Happens model doesn't take seriously enough that sight is a perceptual capacity.

On the Stuff-Happens model, proximal stimulation of sensory transducers sets off a sequence of alterations within the perceiver. At the end of this process, the sensory system is in a certain state. The perceiver with a sensory system in this state enjoys a particular sensory awareness. The terminal state of the sequence of alterations initiated by the proximal stimulation of sensory transducers determines the perceptual experience the perceiver undergoes. So conceived, perceptual experience is a conscious alteration of the perceiving subject.

But the exercise of an capacity is not, or not merely, an alteration. If an object undergoes an alteration it becomes other than what it was. Suppose object \( o \) undergoes some alteration, from \( F \) to \( G \), say, where \( F \) and \( G \) are contraries. Prior to the alteration \( o \) was \( F \) and not \( G \); after the alteration, \( o \) is \( G \) and not \( F \). The exercise of a capacity is crucially different. The exercise of a capacity is the actualization of a potential. Aristotle describes the difference as follows:
\begin{quote}
    Being affected is not a single thing either; it is first a kind of destruction of something by its contrary, and second it is rather the preservation of that which is so potentially by that which is so actually and is like it in the way that a potentiality may be like an actuality. \ldots\ For this reason it is not right to say that something which understands is altered when it understands, any more than a builder when he builds. (Aristotle, \emph{De Anima} \textsc{ii} 5 417\( ^{b} \)\emph{ff}; \citealt[23--24]{Hamlyn:2002ys})
\end{quote}
A builder doesn't cease to be a builder in building a house; rather, he realizes his nature as a builder. To be a builder is to potentially build, and this potentiality is preserved when actually building. Of course, in building a house, the builder will undergo various alterations. When hammering, intrinsic shape changes over time. And it is true that without undergoing some alterations, the builder could not exercise his capacity to build. However, the exercise of a capacity is more than being subject to various alterations. It essentially involves the preservation and actualization of a potentiality.

It is this aspect of capacities and their exercise that the Stuff-Happens model leaves out of account. In maintaining that the terminal state of the sequence of alterations initiated by the proximal stimulation of sensory transducers determines the perceptual experience that the perceiver undergoes, the Stuff-Happens model maintains that the exercise of our perceptual capacities is nothing more than the conscious alteration of the perceiving subject. 


% section the_stuff_happens_model (end)


\section{Capacities and Their Exercise: Existential Dependence} % (fold)
\label{sec:capacities_and_their_exercise}

Sight is a capacity, and seeing is its exercise. What is the relationship between sight and seeing, the capacity and its exercise?

Begin with a flat-footed thought. Sight enables a subject to see. If the subject lacked sight, the subject could not see the scene before them. Lacking sight is an excellent explanation for why a particular subject did not see some salient aspect of the distal environment. The subject's seeing the scene before them seems to depend, in a straightforward way, on the subject's possessing the capacity for sight. Specifically the existence of a given episode of seeing depends on the existence of the subject's capacity for sight. Moreover, this dependency is modal in character: If the subject sees the scene before them, then, necessarily, the subject possesses the capacity for sight. Seeing existentially depends upon sight:
\begin{quote}
    One thing \( x \) will depend upon another \( y \) [in the \emph{existential} sense] just in case it is necessary that \( y \) exists if \( x \) exists (or in the symbolism of modal logic, \( \Box(Ex \rightarrow Ey) \)). \citep[270]{Fine:1995ls} 
\end{quote}
Ricca's smile depends, in this sense, on Ricca. If Ricca's smile exists, then, necessarily, so does Ricca. Thus particularized features or tropes existentially depend on the particulars whose features they are. The definition characterizes a family of existential dependencies with different interpretations of the operative modality---metaphysical or nomological, say---yielding different notions of existential dependence. 

The existential dependence of seeing on sight is an instance of a more general phenomenon, the existential dependence of capacity exercise on capacity possession. Arguably, this is a consequence of the modal character of capacities. A capacity is a kind of potentiality. If a subject has a capacity to do something, they have the potential to do that thing. A subject with a perceptual capacity is a potential perceiver. If, in propitious circumstances, the subject in fact sees something, then, necessarily, it was possible for them to see that thing. (After all, if it wasn't possible for them to see that thing, they wouldn't actually see it.) But this is just the existential dependency of an exercise of a capacity on the capacity whose exercise it is. The existential dependence of capacity exercise on capacity possession is the result of:
\begin{enumerate}[(1)]
    \item Capacities being a species of potentiality, a potential actualized in their exercise
    \item The modal principle: \( p \rightarrow \Box \Diamond p \)
\end{enumerate}

Existential dependence, so characterized, is not asymmetric in the sense of being antisymmetric and irreflexive. There is no logical or a priori reason why there shouldn't be a species of existential interdependence, or if this does not come to the same thing, codependence, obtaining among system of entities, perhaps of correlative kinds. Given this, we can define a stronger notion of existential priority in terms of asymmetric existential dependence:
\begin{quote}
    \( x \) is \emph{existentially prior} to \( y \) \( \leftrightarrow \) \( y \) existentially depends upon \( x \), but \( x \) does not existentially depend upon \( y \).
\end{quote}
The matter of some material compounds are existentially prior to those compounds. If a sand castle exists, then, necessarily, so does the sand that composes it. But the converse is not the case. So sand is existentially prior to sand castles.

Seeing existentially depends upon sight. Is sight existentially prior to seeing? To establish this, it would suffice to show that sight does not existentially depend upon seeing. This amounts to the possibility that a subject possesses the capacity for sight, but there is no episode of seeing which is its exercise. But that is certainly possible (perhaps the hapless subject is blindfolded or unconscious). So not only does seeing existentially depend upon sight, but sight is existentially prior to seeing as well.

The exercise of a capacity existentially depends upon the possession of that capacity. Is a capacity existentially prior to its exercise? The acquisition and persistence of some capacities raises a doubt about this further claim. Consider the capacity to play a musical instrument, the piano, say. Initially, it seems plausible to suppose that if someone is playing the piano then necessarily they have the capacity to play, that playing the piano existentially depends on the capacity to play a piano. However, reflection on what it is to acquire that capacity and for that capacity to persist raises a doubt. 

How is it that a person can come to acquire the capacity to play the piano? Unless one is a preternatural prodigy, one learns to play the piano by, well, playing the piano. One acquires that capacity by repeatedly performing the activity which is its exercise. And if one fails to repeatedly perform the activity which is its exercise, then one would not acquire the capacity to play the piano. But this suggests, that the capacity to play the piano existentially depends upon playing the piano. If possessing the capacity to play the piano existentially depends upon playing the piano, then the capacity to play is not existentially prior to playing. 

This argument is subject to the following criticism: The first stumbling attempts to play, are they really playing? Struggling through a scale or elementary piece of music may count as playing, at least on one standard for playing---keys are moved and sounds issue forth. But such activity is not or not yet the exercise of a capacity. On a more restricted standard for playing that requires playing to be the exercise of a capacity to play, the first stumbling attempts to play don't count as playing. They are not the exercise of a mastery (however limited) of an instrument. At best they are a surrogate for playing engaged in so as to acquire the capacity to genuinely play. But if the activity that the capacity to play existentially depends upon is not itself the exercise of a capacity, then this goes no way towards showing that, the capacity to play is not existentially prior to playing.

Another argument, not subject to this criticism, involves not the acquisition but the persistence of a capacity. For the capacity to play the piano to persist, for it to continue to exist, it must be exercised at more or less regular intervals. Sadly, it is possible to lose this capacity, simply by neglecting to exercise it. But again this suggests that the continued existence of the capacity depends upon the existence of its exercise at regular intervals, that the capacity to play the piano existentially depends upon playing the piano. Since what is at issue is the persistence of the capacity, there is no question that activity that the capacity existentially depends upon is the exercise of that capacity (as there was in the case of its acquisition). But if the capacity to play the piano existentially depends upon playing the piano (understood in the strict sense of an exercise of a mastery of the instrument), then the capacity to play is not existentially prior to playing. This kind of existential interdependence between a capacity and its exercise is inconsistent with the general claim that capacities are existentially prior to their exercise.

\section{Capacities and Their Exercise: Ontological Dependence} % (fold)
\label{sec:capacities_and_their_exercise_ontological_dependence}

Sight enables a subject to see, and so seeing depends for its existence on the capacity of sight. Sight is a sensory capacity that a subject may have or lack. Having the capacity for sight is a way for a subject to be. This way for a subject to be, differs from having the capacity for audition or olfaction, on the one hand, and having non-sensory capacities, such as the capacity to compute prime numbers, on the other. Sight is a way of being, an insight into its nature is gained when we recognize that there is a sense in which sight depends upon seeing.

% section capacities_and_their_exercise_ontological_dependence (end)

Sight enables a subject to see, and so seeing depends for its existence on the capacity of sight. Sight is a sensory capacity that a subject may have or lack. Having the capacity for sight is a way for a subject to be. This way for a subject to be, differs from having the capacity for audition or olfaction, on the one hand, and having non-sensory capacities, such as the capacity to compute prime numbers, on the other. Sight is a way of being, an insight into its nature is gained when we recognize that there is a sense in which sight depends upon seeing.

% section capacities_and_their_exercise (end)

% Bibligography
\bibliographystyle{plainnat} 
\bibliography{Philosophy.bib} 

\end{document}