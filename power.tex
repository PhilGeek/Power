%!TEX TS-program = xelatex 
%!TEX TS-options = -output-driver="xdvipdfmx -q -E"
%!TEX encoding = UTF-8 Unicode
%
%  power.tex
%
%  Created by Mark Eli Kalderon on 2011-11-08.
%  Copyright (c) 2011. All rights reserved.
%

\documentclass[12pt]{article} 

% Definitions
\newcommand\mykeywords{disjunctivism, perception, capacity} 
\newcommand\myauthor{Mark Eli Kalderon} 
\newcommand\mytitle{Experiential Pluralism and the Power of Perception}

\input{preamble/preamble}

%%% BEGIN DOCUMENT
\begin{document}

% Title Page
\maketitle
% \begin{abstract} % optional
% \end{abstract} 
\vskip 2em \hrule height 0.4pt \vskip 2em
% \epigraph{text of epigraph}{\textsc{author of epigraph}} % optional; make sure to uncomment \usepackage{epigraph}

% Layout Settings
\setlength{\parindent}{1em}

% Main Content

\section{Introduction} % (fold)
\label{sec:introduction}

Sight is for the sake of seeing. 

% section introduction (end)

\section{Capacities and Their Exercise} % (fold)
\label{sec:capacities_and_their_exercise}

Sight is a capacity and seeing is its exercise. What is the relationship between sight and seeing, the capacity and its exercise?

Begin with a flat-footed thought. Sight enables a subject to see. If the subject lacked sight, the subject could not see the scene before them. Lacking sight is an excellent explanation for why a particular subject did not see some salient aspect of the distal environment. The subject's seeing the scene before them seems to depend, in a straightforward way, on the subject's possessing the capacity for sight. Specifically the existence of a given episode of seeing depends on the existence of the subject's capacity for sight. Moreover, this dependency is modal in character: If the subject sees the scene before them, then, necessarily, the subject possesses the capacity for sight. In Kit Fine's terminology, the dependence of sight on seeing is modal-existential dependence:
\begin{quote}
    One thing \( x \) will depend upon another \( y \) [in the \emph{modal-existential} sense] just in case it is necessary that \( y \) exists if \( x \) exists (or in the symbolism of modal logic, \( \Box(Ex \rightarrow Ey) \)). \citep[270]{Fine:1995ls} 
\end{quote}
Ricca's smile depends, in this sense, on Ricca. If Ricca's smile exists, then, necessarily, so does Ricca. Thus particularized features or tropes modal-existentially depend on the particulars whose features they are.

Seeing modal-existentially depends upon sight. Is this an instance of a more general phenomena? Does the exercise of a capacity modal-existentially depend upon the capacity whose exercise it is? The acquisition and persistence of some capacities raises a doubt about this more general claim. Consider the capacity to play a musical instrument, the piano, say. Initially, it seems plausible to suppose that if someone is playing the piano then necessarily they have the capacity to play, that playing the piano modal-existentially depends on the capacity to play a piano. However, reflection on what it is to acquire that capacity and for that capacity to persist raises a doubt. How is it that a person can come to acquire the capacity to play the piano? Unless, one is a preternatural prodigy, one learns to play the piano by, well, playing the piano. One acquires that capacity by repeatedly performing the activity which is its exercise. And if one fails to repeatedly perform the activity which is its exercise, then one would not acquire the capacity to play the piano. But this suggests, \emph{contra} our initial supposition, that the capacity to play the piano modal-existentially depends upon playing the piano. Similarly, for the capacity to play the piano to persist, for it to continue to exist, it must be exercised at more or less regular intervals. It is possible to lose this capacity, simply by neglecting to exercise it. But again this suggests, \emph{contra} our initial supposition, that the continued existence of the capacity depends on the existence of its exercise at regular intervals, that the capacity to play the piano modal-existentially depends upon playing the piano.

One line of thought is this: Whether this is a genuine problem depends upon whether modal-existential dependence is asymmetric (in the sense of being antisymmetric and irreflexive). If modal-existential dependence is asymmetric in this sense, then if the capacity to play modal-existentially depends upon playing the piano then this is inconsistent with playing the piano, in turn, modal-existentially depending on the capacity to play. But it is not obvious that modal-existential dependence is asymmetric. It is not obvious that there shouldn't be a species of modal-existential interdependence, or if this does not come to the same thing, codependence obtaining among certain kinds of entities. However, the worry may still persist, since, whether or not modal-existential dependence is asymmetric, it still remains the case that in the first stumbling attempts to play the piano, the initiate lacks the capacity to play. Since practice is temporally prior to the acquisition of the capacity, playing cannot modal-existentially depend on the capacity to play. 

Another line of thought is this: The first stumbling attempts to play, are they really playing? Struggling through a scale or elementary piece of music may count as playing, at least on one standard for playing---keys are moved and sounds issue forth. But such activity is not or not yet the exercise of a capacity. On a more restricted standard for playing that required playing to be the exercise of a capacity to play, the first stumbling attempts to play don't count as playing. They are not the exercise of a mastery (however limited) of an instrument. At best they are a surrogate for playing engaged in so as to acquire the capacity to genuinely play. But if the activity that the capacity to play modal-existentially depends upon is not itself the exercise of a capacity, then this goes no way towards showing that, playing cannot modal-existentially depend upon the capacity to play.

Sight enables a subject to see, and so seeing depends for its existence on the capacity of sight. Sight is a sensory capacity that a subject may have or lack. Having the capacity for sight is a way for a subject to be. This way for a subject to be, differs from having the capacity for audition or olfaction, on the one hand, and having non-sensory capacities, such as the capacity to compute prime numbers, on the other. Sight is a way of being, an insight into its nature is gained when we recognize that there is a sense in which sight depends upon seeing.

% section capacities_and_their_exercise (end)

% Bibligography
\bibliographystyle{plainnat} 
\bibliography{Philosophy.bib} 

\end{document}